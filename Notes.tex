\documentclass{report}

\usepackage[shortlabels]{enumitem}
\usepackage{pgfplots}
\pgfplotsset{compat=1.18}

\input{preamble}
\input{macros}
\input{letterfonts}

\title{\Huge{Algebraic Topology Notes}}
\author{\huge{José Daniel Mejía C.}}
\date{2026}

\begin{document}

\maketitle
\newpage% or \cleardoublepage
% \pdfbookmark[<level>]{<title>}{<dest>}
\pdfbookmark[section]{\contentsname}{toc}
\tableofcontents
\pagebreak

\section{Notation}
\begin{enumerate}
	\item \(\dsone\), The identity map.
	\item \(I\), The interval \([0,1] \in \RR\).
	\item \(\amalg\), Disjoint union of sets or spaces.
	\item \(D^n\), unit disk or ball in \(\RR{n}\).
	\item \(S^n\), unit sphere in \(\RR{n+1}\).
	\item \(\partial D^n = S^{n-1}\), the boundary of the \(n\)-disc.
\end{enumerate}
\pagebreak

\chapter{Geometric notions}

\section{Homotopy and Homotopy Type}

\dfn{Deformation Retraction}{
A \textbf{Deformation Retraction} from a subspace \(X\) onto a subspace \(A\) is a family of maps \(f_t : X \to X, \ t \in I\) such that
\[
f_0 = \dsone, \quad f_1(X) = A \quad \text{and} \quad \restr{f_t}{A} = \dsone \ \forall t \in I
\]

The family \(f_t\) should be continuous in the sense that the associated map \(X \times I \to X, (x,t) \mapsto f_t\) is continuous.
}

A deformation retraction allows us to deform a space onto a subspace of itself continuously. The catch is, this subspace is never affected by the retraction, only \(X \setminus A\) is.

\dfn{Mapping Cylinder}{
For a map \(f : X \to Y\) its \textbf{Mapping Cylinder} \(M_f\), is the quotient space of the disjoint union \(X\times I \amalg Y\), where,

\[
(x,1) \in X\times I \quad \text{identified with} \quad f(x) \in Y
\]
}

This way, the part of \(M_f\) that is \(X\times I\) Has one end deformed to the shape of \(Y\). In this manner \(X\) can deform into \(Y\) as it `moves' through the tube of length 1, as seen in the image. This is properly described as sliding each point \((x,t)\) along the segment \(\{x\} \times I \subset M_f\) to the endpoint \(f(x) \in Y\).

\begin{figure}[h]
	\centering
	\includegraphics[width=0.5\textwidth]{MappingCylinder}
\end{figure}

Not all deformation retractions arise from mapping cylinders though.

\dfn{Homotopy}{
A \textbf{Homotopy} is a family of maps \(f_t : X \to Y, \ t\in I\) such that the associated map \(F: X\times I \to Y, \ F(x,t) = f_t(x)\) is continuous.
Two maps \(f_0, f_1 : X\to Y\) are said to be \textbf{Homotopic} if there is a homotopy \(f_t\) connecting them. it is noted \(f_0 \simeq f_1\) 
}

It becomes clear that a deformation retraction is a special case of a homotopy. It is in fact, a homotopy that connects the identity map on \(X, \dsone = f_0\) to its retraction on \(A\), a map \(r: X\to X\) such that \(r(X) = A\) and \(\restr{r}{A} = \dsone\).

Retractions are the topological analogs of projection operators in other parts of mathematics.

\dfn{Homotopy relative to \(A\)}{
A homotopy is said to be \textbf{relative to} \(A \subset X\) if its restriction to \(A\) is independent of \(t\).
}

For example, a deformation retraction of \(X\) onto \(A\) is a homotopy \textbf{rel} \(A\) from the identity map of \(X\) to a retraction of \(X\) onto \(A\), since \(\restr{f_t}{A} = \dsone\), independent of \(t\).

\dfn{Homotopy Equivalent Spaces}{
A map \(f:X \to Y\) is a \textbf{Homotopy Equivalence} if there exists a map 
\[
g: Y \to X \quad \text{such that} \quad fg \simeq \dsone \ \text{and} \ gf \simeq \dsone
\]


In this case, the spaces \(X\) and \(Y\) are said to be \textbf{Homotopy Equivalent} or to have the same \textbf{Homotopy Type}. Notation is \(X \simeq Y\).
}

This way, for example, several deformation retractions of the same space are homotopy equivalent; despite not being deformation retractions of each other necessarily.

Homotopy equivalence is an \textbf{equivalence relation}.

\sep

A space having the homotopy type os a single point is \textbf{Contractible}. This implies the identity map of this space is \textbf{Nullhomotopic}, meaning it's homotopic to a constant map. The converse is not true.

\section{Cell Complexes}

\dfn{Genus}{
The genus of a connected, orientable surface \(M_g\) is the maximum number of cuttings along non-intersecting closed simple curves without rendering the resultant manifold disconnected.It is equal to the number of handles (or `holes') on it.
}

An orientable surface \(M_g\) of genus \(g\) can be constructed using a polygon with \(4g\) sides and identifying their edges pairwise.

\fgr{Genus}{0.4}

\dfn{\(n\)-cells}{
An \(n\)-cell, at least in \(\RR[n]\) are discs \(D^n\). These are \(n\)-dimensional spaces that can be glues together along their boundaries.
}

The torus can be built by first gluing two \(1\)-cells together, \(a\) and \(b\), resulting in a kind of `wire' skeleton. Then, gluing to it a \(2\)-cell (the interior of the polygon).

\dfn{Cell Complex}{
A \textbf{Cell Complex} is a space \(X\) constructed by the following procedure:

\begin{enumerate}
	\item Start with the discrete set \(X^0\), whose points are regarded as \(0\)-cells. 
	\item Inductively, form the \textbf{\(n\)-skeleton} \(X^n\) from \(X^{n-1}\) by attaching \(n\)-cells \(e^n_\alpha\) via maps \(\varphi_\alpha : S_{n-1} \to X^{n-1}\). \\ This means \(X_n\) is the quotient space of the disjoint union \(X^{n-1} \amalg_\alpha D^n_\alpha\) of \(X^{n-1}\) with a collection of \(n\)-disks \(D_\alpha^n\) with the identifications \(x \sim \varphi_\alpha(x)\) for \(x\in \partial D^n_\alpha\). Thus as a set, \(X^n = X^{n-1} \amalg_\alpha e^n_\alpha\) where each \(e^n_\alpha\) is an open \(n\)-disk.
	\item The process can be stopped at a finite number of steps, given \(X = X^n\) for an \(n < \infty\). If not, \(X = \bigcup_n X^n\). Here \(X\) is given the weak topology, meaning a set \(A \subset X\) is open (or closed), iff \(A \cap X^n\) is open (or closed) in \(X^n\) for each \(n\).
\end{enumerate}
}

If \(X = X^n\) for some \(n\), \(X\) is said to be finite-dimensional, and the smallest of such \(n\) is the \textbf{Dimension} of \(X\), the maximum dimension of cells in \(X\).

\ex{\(1\) cell complex}{
A \(1\) cell complex is a \textbf{Graph}, consisting of \(0\)-cells, (vertices) and \(1\)-cells (edges).
}

\dfn{Euler Characteristic}{
The \textbf{Euler Characteristic} of a cell complex is the number of even-dimensional cells minus the number of odd-dimensional cells.

The Euler characteristic of a cell complex depends only of its homotopy type.
}

\ex{Sphere \(S_n\)}{
The sphere \(S^n\) has a structure with only two cells, \(e^0\) and \(e^n\), \(e^0\) being its center point and \(e^n\), the \(n\)-cell being attached by the constant map \(S^{n-1} \to e^0\).

This is equivalent to regarding \(S^n\) as the quotient space \(D / \partial D^n\), as in folding all the edge of a disk onto a single point.
}

\dfn{Characteristic Map}{
Each cell \(e_\alpha^n\) in a cell complex \(X\) has a \textbf{Characteristic Map} \(\Phi_\alpha : D^n_\alpha \to X\), which extends the attaching map \(\varphi_\alpha\) and is a homeomorphism from the interior of \(D^n_\alpha\) onto \(e_\alpha^n\)
}

The characteristic map homeomorphically maps the interior of the disk onto an open \(n\)-cell \(e^{n}\) and maps the boundary \(\partial D^{n}=S^{n-1}\) into the \((n-1)\)-skeleton \(X^{n-1}\).

The attaching map defines how the boundary of a new \(n\)-cell is glued onto the existing skeleton (\(X^{n-1}\)), while the characteristic map defines the entire cell's position and structure within the space.
 
\ex{Characteristic map of \(S^n\)}{
With \(S^n\) built as a CW-complex as presented above, the \textbf{characteristic map} for the \(n\)-cell is the quotient map \(D^n \to S^n\), collapsing \(\partial D^n\) to a point.
}

\dfn{Subcomplex}{
A \textbf{Subcomplex} of a cell complex \(X\) is a closed subspace \(A \subset X\) that is a union of cells in \(X\).
}

A subcomplex is a cell complex in itself, since \(A\) being closed implies that each characteristic map of it cells has image contained in \(A\).

A pair \(X,A\), consisting of a cell complex and a subcomplex is called a \textbf{CW pair}.

Each skeleton \(X^n\) is a subcomplex of the cell complex \(X\).

\nt{Cell complex constructions for an object are not unique. There are infinitely many ways to arrange cells in order to build a space. In this sense, sometimes a cell complex can or cannot be a subcomplex of another. It depends on the structure given initially for both.}


 
\end{document}
