\documentclass{report}

\usepackage[shortlabels]{enumitem}
\usepackage{pgfplots}
\pgfplotsset{compat=1.18}

\input{preamble}
\input{macros}
\input{letterfonts}

\title{\Huge{Algebraic Topology Notes}}
\author{\huge{José Daniel Mejía C.}}
\date{2026}

\begin{document}

\maketitle
\newpage% or \cleardoublepage
% \pdfbookmark[<level>]{<title>}{<dest>}
\pdfbookmark[section]{\contentsname}{toc}
\tableofcontents
\pagebreak

\section{Notation}
\begin{enumerate}
	\item \(\dsone\), The identity map.
	\item \(I\), The interval \([0,1] \in \RR\)
	\item \(\amalg\), Disjoint union of sets or spaces.
\end{enumerate}
\pagebreak

\chapter{Geometric notions}

\dfn{Deformation Retraction}{
A \textbf{Deformation Retraction} from a subspace \(X\) onto a subspace \(A\) is a family of maps \(f_t : X \to X, \ t \in I\) such that
\[
f_0 = \dsone, \quad f_1(X) = A \quad \text{and} \quad \restr{f_t}{A} = \dsone \ \forall t \in I
\]

The family \(f_t\) should be continuous in the sense that the associated map \(X \times I \to X, (x,t) \mapsto f_t\) is continuous.
}

A deformation retraction allows us to deform a space onto a subspace of itself continuously. The catch is, this subspace is never affected by the retraction, only \(X \setminus A\) is.

\dfn{Mapping Cylinder}{
For a map \(f : X \to Y\) its \textbf{Mapping Cylinder} \(M_f\), is the quotient space of the disjoint union \(X\times I \amalg Y\), where,

\[
(x,1) \in X\times I \quad \text{identified with} \quad f(x) \in Y
\]
}

This way, the part of \(M_f\) that is \(X\times I\) Has one end deformed to the shape of \(Y\). In this manner \(X\) can deform into \(Y\) as it `moves' through the tube of length 1, as seen in the image. This is properly described as sliding each point \((x,t)\) along the segment \(\{x\} \times I \subset M_f\) to the endpoint \(f(x) \in Y\).

\begin{figure}[h]
	\centering
	\includegraphics[width=0.5\textwidth]{MappingCylinder}
\end{figure}

Not all deformation retractions arise from mapping cylinders though.

\dfn{Homotopy}{
A \textbf{Homotopy} is a family of maps \(f_t : X \to Y, \ t\in I\) such that the associated map \(F: X\times I \to Y, \ F(x,t) = f_t(x)\) is continuous.
Two maps \(f_0, f_1 : X\to Y\) are said to be \textbf{Homotopic} if there is a homotopy \(f_t\) connecting them. it is noted \(f_0 \simeq f_1\) 
}

It becomes clear that a deformation retraction is a special case of a homotopy. It is in fact, a homotopy that connects the identity map on \(X, \dsone = f_0\) to its retraction on \(A\), a map \(r: X\to X\) such that \(r(X) = A\) and \(\restr{r}{A} = \dsone\).

Retractions are the topological analogs of projection operators in other parts of mathematics.

\dfn{Homotopy relative to \(A\)}{
A homotopy is said to be \textbf{relative to} \(A \subset X\) if its restriction to \(A\) is independent of \(t\).
}

For example, a deformation retraction of \(X\) onto \(A\) is a homotopy \textbf{rel} \(A\) from the identity map of \(X\) to a retraction of \(X\) onto \(A\), since \(\restr{f_t}{A} = \dsone\), independent of \(t\).

\dfn{Homotopy Equivalent Spaces}{
A map \(f:X \to Y\) is a \textbf{Homotopy Equivalence} if there exists a map 
\[
g: Y \to X \quad \text{such that} \quad fg \simeq \dsone \ \text{and} \ gf \simeq \dsone
\]


In this case, the spaces \(X\) and \(Y\) are said to be \textbf{Homotopy Equivalent} or to have the same \textbf{Homotopy Type}. Notation is \(X \simeq Y\).
}

This way, for example, several deformation retractions of the same space are homotopy equivalent; despite not being deformation retractions of each other necessarily.

Homotopy equivalence is an \textbf{equivalence relation}.

\sep

A space having the homotopy type os a single point is \textbf{Contractible}. This implies the identity map of this space is \textbf{Nullhomotopic}, meaning it's homotopic to a constant map. The converse is not true.


 
\end{document}
